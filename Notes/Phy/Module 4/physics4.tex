\documentclass[12pt, a4paper]{article}
\usepackage[a4paper, top=2cm, bottom=3cm, left=2cm, right=2cm]{geometry}
\usepackage[export]{adjustbox}
\usepackage{graphicx}
\usepackage{mathtools}
\usepackage{hyperref}
\usepackage{amsmath}
\usepackage{amsfonts}
\usepackage{amssymb}
\usepackage[version=4]{mhchem}
\usepackage{stmaryrd}
\usepackage{polyglossia}
\usepackage{fontspec}
\usepackage{ucharclasses}
\usepackage{fancyhdr}
\usepackage{wrapfig}
\usepackage{subcaption}
\usepackage{relsize}
\usepackage{makecell}
\usepackage{framed}
\usepackage{changepage}
\usepackage{tabularray}
\usepackage{etoolbox}
\usepackage{xstring}
\usepackage{pstricks-add}
\usepackage{tikz}
\usepackage{empheq}
\usepackage[skins]{tcolorbox}
\usepackage[european,s traightvoltages, americanresistor, americaninductors]{circuitikz}
\usepackage{pgfplots}
\usepackage{tikz-3dplot}
\usepackage[T1]{fontenc}
\usepackage[en-GB]{datetime2}
\usepackage{marginnote}
\usepackage{tikzpagenodes}

\usetikzlibrary{
	angles,
	arrows,
	arrows.meta,
	backgrounds,
	calc,
	decorations,
	decorations.markings,
	decorations.pathmorphing,
	fit,
	patterns,
	positioning,
	quotes,
	shapes.arrows,
	shapes.callouts,
	shapes.geometric,
	shapes.misc,
	snakes,
}

\pgfplotsset{compat=1.18}

\hypersetup{colorlinks=true, linkcolor=blue, filecolor=magenta, urlcolor=cyan,}
\urlstyle{same}

\setmainlanguage{english}
\setotherlanguages{norwegian, arabic}
\newfontfamily\arabicfont{Noto Naskh Arabic}
\graphicspath{ {../images/} }

\renewcommand\theadfont{\bfseries}
\DTMlangsetup[en-GB]{abbr}


%%%%%%%%%% Fancy header %%%%%%%%%
\pagestyle{fancy}
\fancyhead[C]{}
\fancyfoot[C]{\medskip\thepage}
\renewcommand{\footrulewidth}{.4pt}
\renewcommand{\headrulewidth}{0pt}
\setlength{\headheight}{14.49998pt}
\addtolength{\topmargin}{-2.49998pt}
\newcommand{\figwidth}{8cm}
\newcommand{\floatfigwidth}{5cm}



%%%%%%%%%% constants/symbols %%%%%%%%%
\newcommand\longUparrow{\mathrel{\scalebox{1}[2]{$\uparrow$}}}
\DeclareRobustCommand{\rchi}{{\mathpalette\irchi\relax}}
\newcommand{\irchi}[2]{\raisebox{\depth}{$#1\chi$}}
\newcommand{\term}[1]{\underline{\textbf{#1}}}
\newcommand{\amstr}{\mathring{\textrm{A}}}
\newcommand{\h}{6.626 \times 10^{-34}}
\newcommand{\kB}{1.38 \times 10^{-23}}
\newcommand{\lc}{3 \times 10^{8}}
\newcommand{\uunit}[1]{\mathrm{~#1}}
\newcommand{\doparindent}{\setlength\parindent{.5cm}}
\newcommand{\noparindent}{\setlength\parindent{0pt}}



\DefTblrTemplate{caption-tag}{default}{}
\DefTblrTemplate{caption-sep}{default}{}
\DefTblrTemplate{caption-text}{default}{}
\DefTblrTemplate{contfoot-text}{default}{}
\DefTblrTemplate{conthead-text}{default}{}

\noparindent
\newcommand{\uprimary}[1]{
	\section*{\center \Huge \underline{#1}}
	\addcontentsline{toc}{section}{\protect\numberline{}#1}
}
\newcommand{\usecondary}[1]{
	\section*{\center \LARGE \underline{#1}}
	\addcontentsline{toc}{section}{\protect\numberline{}#1}
}
\newcommand{\topic}[1]{
	\section*{\LARGE \fontfamily{ppl}\selectfont#1}
	\addcontentsline{toc}{subsection}{\protect\numberline{}#1}
}
\newcommand{\subtopic}[1]{
	\section*{\Large \fontfamily{ppl}\selectfont #1}
	\addcontentsline{toc}{subsection}{\protect\numberline{}#1}
}
\newcommand{\ussubsection}[1]{
	\section*{\large \fontfamily{ppl}\selectfont#1}
	\addcontentsline{toc}{subsection}{\protect\numberline{}#1}
}
\newcommand{\ans}{\bigskip\underline{\textbf{Answer}}}
\newcommand{\ques}[1]{\noparindent\textbf{#1}\doparindent}
\newcommand{\rfloatingimg}[1]{
	\begin{wrapfigure}{r}{\floatfigwidth}
		\includegraphics[max width=\floatfigwidth]{#1}
	\end{wrapfigure}
}
\newcommand{\indentbox}[2]{
	\begin{adjustwidth}{#1}{0pt}
		#2
	\end{adjustwidth}
}
\newcommand{\qa}[3]{
	\noparindent
	\textbf{#1 #2}
	\indentbox{.76cm}{
		\ans
		#3
	}
	\vspace{.75cm}
}
\newcommand{\noskipqa}[2]{
	\noparindent
	\textbf{#1}
	\indentbox{.76cm}{
		\ans
		#2
	}
}
\newcommand{\eqnleft}[1]{
	\begin{flalign*}
		 & #1 &  &
	\end{flalign*}
}
\newcommand{\fullwidthimg}[1]{
	\begin{center}
		\includegraphics[max width=\textwidth]{#1}
	\end{center}
}
\newcommand{\umodule}[2]{
	\uprimary{Module - #1}
	\vspace{-.7cm}
	\usecondary{#2}
}

% {note header}{width}{actual note}
\newcommand{\note}[3]{
	\begin{tcolorbox}[
			width=#2,
			left=5pt,
			colframe=blue!50!black!70!white,
			colback=blue!10,
			title={\textbf{#1}}
		]
		#3
	\end{tcolorbox}
}
% {width}{content}[framecolor][bgcolor]
\NewDocumentCommand{\sidebox}{m m O{orange!80!black!90} O{orange!10}}{
	\begin{tcolorbox}[
			width=#1,
			left=5pt,
			colframe=#3,
			colback=#4
		]
		#2
	\end{tcolorbox}
}

\NewDocumentCommand{\multiskip}{m}{%
	\begingroup
	\newcount\i  % Define a new counter \i
	\i=0         % Initialize the counter
	\loop
	\ifnum\i<#1
	\bigskip  % Add \bigskip
	\advance\i by 1  % Increment the counter
	\repeat
	\endgroup
}


\newcommand{\termlist}[1]{
	\begin{tcolorbox}[
			colback=blue!10!white,
			colframe=blue!50!black,
			title={Some terms}
		]
		#1
	\end{tcolorbox}
}

\newcommand{\chapterheader}[2]{
	\resizebox{\textwidth-3pt}{!}{
		\begin{tikzpicture}[very thick]
			\node [
				draw=black,
				minimum height= 3cm,
				minimum width = \textwidth
			] at (.5\textwidth, 1.5) {
				\scshape 
				\parbox{\textwidth}{\fontsize{28pt}{28pt}\selectfont{}\centering #2}
			};
			\node[fill=white, minimum width = 1.3cm] at (.5\textwidth, 3) {\Huge #1};
		\end{tikzpicture}
	}
	\vspace*{.7cm}
}

\begin{document}

\chapterheader{4}{Magnetism}

Magnetism arises from two types of motion of electrons in an atom:
\begin{itemize}
	\item The motion of electrons in an orbit around the nucleus (similar to motion of planets in one solar system around the sun).
	\item Spin of electron around its own axis (analagus to rotation of it about its own axis).
\end{itemize}
\doparindent
These two motions impart to magnetic moment of electron causing each of them to behove as a ting magnet. The region in space that is penetrated by the imaginary lines of magnetic force describes a magnetic field. The strength of magnetic field is determined by the number of lines of force per unit area of space. magnetic fields are created on a large scale, either by the passage of an electric current through magnetic metals or by magnetised materials called magnets.

The portion of this space in the neighbarhoot of a magnetised body over which it excerts its influence is called magnetic field.

If the total of line of force comprising the magnetic field is called magnetic flux ($\phi$) magnetic flux density is defined as flux passing per unit area in a material through a plane at the right angles to the flux
$$B=\phi/A$$
\noparindent
It is also proportional to intensity of magnetic field
$$
	B=\mu H$$

Where $\mu=$ permeability and it depends on the magnetic properties.

In the absence of any medium,
$$B=\mu_0 H$$

Where $\mu_0$ =absolute permeability ($\mu_0 = 4\pi\times10^7\ \mathrm{H/m}$)

The ratio $\mu/\mu_0=\mu_r$ is called relative permeability
$$
	B=\mu_0 \mu_r H
$$

The origin of magnetism lies in the orbital and spin motion of electrons and how the electrons interact with one another end how materials respond to magnetic field. In some materials there is no collective interaction of atomic magnetic moments, whereas in other materials there is very strong interaction between atomic moments.

The magnetic behaviour of materials can be classified into following five groups.
- Diamagnetic, Paramagnetic, ferrimagnetic and anti-Ferromagnetic

\subtopic{Diamagnetism}
Diamagnetic materials are slightly repelkd by a magnetic field and the material does not retain the magnetic properties when external field is removed. It arises from the realignment of electron orbits under the influence of external magnetic field. Most elements including Cu, Ag, Au are diamagnetic. Diamagnetic materials have a very weak and negative susceptibility ($\rchi$).
The atomic number of diamagnetic material materials, is generally an even number and hence net magnetic moment of each atom is zero in diamagnetic material. When a diamagnetic material is placed is a external magnetic field, it gets weakely magnetised in the direction opposite to that of external field.

Relative permeability of diamagnotic maternal is $<1$. susceptibility is independent of temperature.

\subtopic{Paramagnetism}
In Paramagnetic materials some of the atoms in the material have net magnetic moments due to unpaired electrons in partially filled orbitals. Paramagnetic properties are due to presence of these unpaired electrons and from the realignment of electron orbits caused by external magnetic field. They are slightly attracted by magnetic field and material doesn't retain the magnetic properties when external field is removed. In the presence of field there is a partial alignment of atomic magnetic moments in the direction of the fill, resulting in a net magnetisation. Hence they have positive susceptibility. Examples are: Molybdinum, Lithium, Tantalum. The atomic number of paramagnetic materials is gerally odd number. Hence net magnetic moment of each atom is non-zero. But these atomic magnetic moments are oriented randomly. Inside the material and hence the overall magnetic moment of material is negligible. The relative permeability is greater than 1 and susceptibility varies inversely with temperature. $$\uparrow \downarrow \uparrow \uparrow \downarrow \uparrow$$

\subtopic{Ferromagnetism}
These materials have large and positive susceptibility to an external magnetic field. They exhibit strong attraction to magnetic field and are able to retain their magnetic properties even aftor the external field has been removed. These have some unpaired electrons and therefore their atoms have net magnetic moment, e.g: Fe, Ni, Co. They get their strong magnetic properties owing to their presence of magnetic domains. In these domains, large number of atoms are aligned parallel to each other. So that magnetic force within the domain is strong. When a ferromagnetic material is in the unpaired state the domains are nearly randomly organised and net magnetic field, for the part as a whole is zero.

$$\uparrow \uparrow \uparrow \uparrow \uparrow$$

\subtopic{Ferrimagnetism}
As a result of crystal structure, in some material a complex form of magnetic ordering can also be possible. Here the magnetic structure is composed of two magnetic sublattices ($A \& B$). The sublattice $A$ has atoms with spin moments oriented in one direction, and sublattice $B$ has atoms with spin moments in opposite direction.
The magnetic moments of sublattices $A \& B$ are not equal, resulting in a net magnetic onoment. So ferrimagnetic materials are therefore similar to ferro magnetic materials, e.s: $\mathrm{FeO}, \mathrm{Fe}_2 \mathrm{O}_3, \cdots$


$$
	\longUparrow \downarrow \longUparrow \downarrow \longUparrow \downarrow \longUparrow \downarrow
$$

\subtopic{Anti-ferromagnetism}
In some crystals the alternate atoms have the ir spins parallel to each other but not the adjacent atom. Such crystals can be considered to be composed of two types of sublattices ($A \& B$). $A$ has atoms with their spin moment oriented in one direction and $B$ has atoms with their spin moments oriented in opposite direction. As the $A \& B$ sublattice moments are exactly equal and opposite, the net magnetic moment is zero.
e.g: FeS.
$$\uparrow \downarrow \uparrow \downarrow \uparrow \downarrow$$

\subtopic{Magnetic Materials}
A magnetic material is a type of substance which gets attracted towards a magnet. The magnetic material also acquires magnetisation when it is placed in a magnetic field of a magnet. Depending on the magnetisation propertius there are 2 types of magnetic matoials: Hard a soft mag moderials.

Hard magnetic materials, (aka permanent magnets) are materials with high coercivity. and strong magnetisation that can retain their magnetisation even in the absence of external magnetic field. These materials have a high resistance to demagnetisation and require substantial external magnetic field to change their magnetic orientation, (e.g: Neodymium, NdFeB, Sm, Co). These materials are widely used in applications such as electric motors speakers, magnetic storage devices and sensors. Their ability to maintain a permanent magnetic field makes them valuable in technologies where a strong and stable magnetic field is required.

Features of Hard magnetic materials are:
\noparindent

\textbf{high coevcivity}: Hard magnetic exhibit high coevcivity, means that they require significant expex external magnetic field to change their magnetic orientation. This characteristics enables them to retain their magnetisation even in the absence of an external field. making them ideal for permanent magnet applications.

\textbf{Strong magnetisation}: There materials have high saturation magnetisation allowing them to generate strong magnetic fields. This property makes them valuable in applications where a powerful and stable magnetic field is required (such as in electric motors or MRI).

\textbf{Long term stability}: These materials have excellent thermal stability. They maintaining their magnetic properties at high temperatures. This characteristre is cruicial in automotive and aerospace industries.

\textbf{Resistance to demagnetisation}: These materials have strong resistance to demagnetisation, meaning they can retain their magnetisation over extended periods without significant loss. This property ensures their long therm functionality and reliability in applications like magnetic storage devices.


% 16 - 23, 14, 7, 1, 2

%10
\end{document}

