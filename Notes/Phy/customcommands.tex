\usepackage[utf8]{inputenc}
\usepackage[a4paper, top=2cm, bottom=3cm, left=2cm, right=2cm]{geometry}
\usepackage[export]{adjustbox}
\usepackage{graphicx}
\graphicspath{ {./images/} }
\usepackage{mathtools}
\usepackage{amsmath}
\usepackage{amsfonts}
\usepackage{amssymb}
\usepackage[version=4]{mhchem}
\usepackage{stmaryrd}
\usepackage[fallback]{xeCJK}
\usepackage{polyglossia}
\usepackage{fontspec}
\usepackage{ucharclasses}
\usepackage{fancyhdr}
\usepackage{wrapfig}
\usepackage{subcaption}
\usepackage{relsize}
\usepackage{framed}
\usepackage{changepage}
\usepackage{tabularray}
\usepackage{etoolbox}
\usepackage{xstring}

\setCJKmainfont{Noto Serif CJK TC}

\setmainlanguage{english}
\setotherlanguages{norwegian, arabic}
\newfontfamily\arabicfont{Noto Naskh Arabic}
% \newfontfamily\lgcfont{CMU Serif}

\pagestyle{fancy}
\fancyhead[C]{}
\fancyfoot[C]{\medskip\thepage}
\renewcommand{\footrulewidth}{.4pt}
\renewcommand{\headrulewidth}{0pt}

\setlength{\headheight}{14.49998pt}
\addtolength{\topmargin}{-2.49998pt}

\newcommand{\figwidth}{8cm}
\newcommand{\floatfigwidth}{5cm}



\newcommand{\uprimary}[1]{
	\section*{\center \Huge \underline{#1}}
	\addcontentsline{toc}{section}{\protect\numberline{}#1}
}
\newcommand{\usecondary}[1]{
	\section*{\center \LARGE \underline{#1}}
	\addcontentsline{toc}{section}{\protect\numberline{}#1}
}
\newcommand{\usection}[1]{
	\section*{\LARGE #1}
	\addcontentsline{toc}{subsection}{\protect\numberline{}#1}
}
\newcommand{\usubsection}[1]{
	\section*{\Large #1}
	\addcontentsline{toc}{subsection}{\protect\numberline{}#1}
}
\newcommand{\ussubsection}[1]{
	\section*{\large #1}
	\addcontentsline{toc}{subsection}{\protect\numberline{}#1}
}
\newcommand{\ans}{\bigskip\underline{\textbf{Answer}}}
\newcommand{\rfloatingimg}[1]{
	\begin{wrapfigure}{r}{\floatfigwidth}
		\includegraphics[max width=\floatfigwidth]{#1}
	\end{wrapfigure}
}
\newcommand\longUparrow{\mathrel{\scalebox{1}[2]{$\uparrow$}}}

\newcommand{\doparindent}{\setlength\parindent{.5cm}}
\newcommand{\noparindent}{\setlength\parindent{0pt}}
\newcommand{\ques}[1]{\noparindent\textbf{#1}\doparindent}

\newcommand{\indentbox}[2]{
	\begin{adjustwidth}{#1}{0pt}
		#2
	\end{adjustwidth}
}
\newcommand{\qa}[3]{
	\noparindent
	\textbf{#1 #2}
	\indentbox{.76cm}{
		\ans
		#3
	}
	\bigskip
}
\newcommand{\noskipqa}[2]{
	\noparindent
	\textbf{#1}
	\indentbox{.76cm}{
		\ans
		#2
	}
}

\newcommand{\eqnleft}[1]{
	\begin{flalign*}
		 & #1 &  &
	\end{flalign*}
}
\DeclareRobustCommand{\rchi}{{\mathpalette\irchi\relax}}
\newcommand{\irchi}[2]{\raisebox{\depth}{$#1\chi$}}
\newcommand{\term}[1]{\underline{\textbf{#1}}}
\newcommand{\amstr}{\mathring{\textrm{A}}}
\newcommand{\h}{6.626 \times 10^{-34}}
\newcommand{\kB}{1.38 \times 10^{-23}}
\newcommand{\lc}{3 \times 10^{8}}
\newcommand{\unit}[1]{\mathrm{~#1}}
\newcommand{\fullwidthimg}[1]{
	\begin{center}
		\includegraphics[max width=\textwidth]{#1}
	\end{center}
}
\newcommand{\uheading}[2]{
	\uprimary{Module - #1}
	\vspace{-.7cm}
	\usecondary{#2}
}

\newcommand{\note}[1]{
	\begin{framed}
		\textbf{Note:}
		\medskip
		\indentbox{.6cm}{
			#1
		}
	\end{framed}
}


% Define the new command with looping functionality
\NewDocumentCommand{\multiskip}{m}{%
	\begingroup
	\newcount\i  % Define a new counter \i
	\i=0         % Initialize the counter
	\loop
	\ifnum\i<#1
	\bigskip  % Add \bigskip
	\advance\i by 1  % Increment the counter
	\repeat
	\endgroup
}

\DefTblrTemplate{caption-tag}{default}{}
\DefTblrTemplate{caption-sep}{default}{}
\DefTblrTemplate{caption-text}{default}{}
\DefTblrTemplate{contfoot-text}{default}{}
\DefTblrTemplate{conthead-text}{default}{}
