\bigskip

\begin{center}
\begin{tikzpicture}[thick]
	\draw (0, 0) -- (0,3.3);
	\draw[color=gray] (7.5, 0) -- (7.5,3.3);
	\foreach \i in {0,...,15} {
			\draw (0, \i/15*3.3) --++(-.4, -.3);
			\draw[color=gray] (7.5, \i/15*3.3) --++(.4, -.3);
		}

	\draw (1.75,.5) rectangle (5.75,2.75);
	\node[align=center] at (1.75+2,.5+1.125) {Active \\ Medium};
	
	\node at (7.5/2, 4.5) {Pumping process};
	\node[align=center] at (-.3, -1.3) {100\% \\ reflective};
	\node[align=center] at (7.7, -1.3) {partially \\ reflective};
	\node[align=center,rotate=90] at (11, 1.7) {Laser};
	
	\foreach \i in {1, ..., 3}{
		\draw[-{Latex[length=3mm]}] (0, 3/4*\i+.15) -- ++(1.4, 0);
		\draw[-{Latex[length=3mm]}] (7.5, 3/4*\i+.15) -- ++(-1.4, 0);
		\draw[-{Latex[length=3mm]}] (8.3, 3/4*\i+.15) -- ++(+2.4, 0);
		\draw[-{Latex[length=3mm]}] (7.5/2 - 6/4 + 3/4*\i, 4) -- ++(0, -1);
		}

\end{tikzpicture}
\end{center}
\bigskip
\termlist{
	\term{Active medium}: material in which popalation inversion takes place. (eg. ruby, Ga-In, He-Ne)

\term{Active center}: Atoms which can produce more stimulated emission than spontaneous emission: (e.g: Cr in ruby, Ne in He-Ne medium).

\term{Pumping mechanism}: Adding energy to system. This can be done by:
\begin{itemize}
	\item Optical pumping (used in ruby laser, xe-flash lamp)
	\item Direct electron excitation.
	\item Inelastic atom-atom collisions: electric discharge is employed to cause collision and excitation of atoms. (used in He-Ne)
	\item Chemical reactions.

\end{itemize}

}
\ussubsection{Optical resonance \& Resonance cavity}

Two mirrors are used to reflect light back into active medium one will be 100 \% reflective.(Usually made of dielectic material) and other one is partially reflective.

Two arrangements of mirrors are used to reflect light:
\begin{itemize}
	\item \textbf{plane-parallel}: needs to be strictly parallel (with deviation less than 1" (1/3600$^{\deg}$)) and must have smooth surface(irregularities <1/100$\lambda$).
	\item \textbf{confocal}: need to be parallel (with deviation less than 0.025$^{\deg}$)
\end{itemize}


\ussubsection{Types of Laser}

\begin{itemize}
	\item \textbf{Pulsed mode}: Train of pulses are produced. It can have produce power in the order of 1MW. (e.g: Ruby laser)
	\item \textbf{Continuous mode}: Light is produced continuously. It has less powerful output (less than 1W) (eg: He-Ne laser)
\end{itemize}

\ussubsection{Uses of Laser}

\begin{itemize}
	\item Used in holograms
	\item To study the internal structure of molecules.
	\item To detect earthquakes
	\item For cutting gilling and welding
	\item Used to perform surgery
	\item Used to treat cancer, kidncy stone, tumers etc.
	\item Used to detect and destroy enemy missiles and also to control rockets and satellites.
\end{itemize}