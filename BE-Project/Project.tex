\documentclass[12pt, a4paper]{article}
\usepackage{geometry}
\usepackage[utf8]{inputenc}
\usepackage{amsmath}
\usepackage{amsfonts}
\usepackage{mathrsfs}
\usepackage{fancybox,graphicx}
\usepackage{color}
\usepackage[colorlinks]{hyperref}
\usepackage{accents}
\usepackage{cite}
\usepackage{svg}
\usepackage{fancyhdr}
\usepackage{minted}

\pagestyle{fancy}
\fancyhead[C]{}
\fancyfoot[C]{} % quick fix
\fancyfoot[R]{\thepage}
\addtolength{\headwidth}{\marginparsep}
\addtolength{\headwidth}{\marginparwidth}
\renewcommand{\headrulewidth}{0pt}
\addtolength{\topmargin}{-4.0pt}
\setlength{\headheight}{14.49998pt}

\begin{document}
\begin{titlepage}
	\newgeometry{top=2.5in, bottom=1in, left=1in, right=1in}
	\centering

	{\huge \textbf{Visualisation of Audio using\\ Dot Matrix Display} \par}

	\vspace{1.3cm}

	{\large Submitted by \par}

	\vspace{.5cm}

	{\large \textbf{Roopesh O R}\par}
	{\large \textbf{Prabhath C S}\par}
	{\large \textbf{Pranav Praveen}\par}

	\vspace{3cm}

	\includesvg[inkscapearea=drawing, width=1.5in]{cusat.svg}\par

	\vspace{.1cm}

	{\textbf{Division of Electronics Engineering} \par}
	{\textbf{School of Engineering} \par}
	{\textbf{Cochin University of Science and Technology} \par}
	{\textbf{Kochi - 682022} \par}

	\vspace{.5cm}

	{\textbf{June 2024} \par}

	\vfill

\end{titlepage}

\newgeometry{top=3cm, bottom=3cm, left=2cm, right=2cm}

\newcommand{\usection}[1]{
	\section*{\center \Huge #1}
	\addcontentsline{toc}{section}{\protect\numberline{}#1}
}
\newcommand{\usubsection}[1]{
	\section*{\LARGE #1}
	\addcontentsline{toc}{subsection}{\protect\numberline{}#1}
}

\vspace*{4cm}
\usection{Abstract}
\vspace{.5cm}
From simple music player applications to concerts, audio
visualization enhances the ambience and overall experience
of the enjoyer. In this project, we replicate a simpler
version of such systems using an Arduino UNO development
board, 32*8 LED dot matrix display, and a few small components.
An audio signal of our interest is preprocessed and fed
to one of the analogue pins of Arduino. This signal is then
processed using the "ArduinoFFT" library, which splits the
audio into discrete chunks and performs FFT on it. The
resulting frequency spectrum is then again processed and
scaled down to match the width and height of the LED display.
Finally, the resulting waveform is displayed. In the end,
we try to add more customizations such as different "display modes" and effects.


\pagebreak

\usection{Implementation}
\par
\vspace*{1cm}
\usubsection{Introduction}
dfdsfsas

\vspace*{1cm}
\usubsection{Block Diagram}
dggds

\newpage
\usubsection{Arduino UNO}
\vspace{1cm}
\begin{center}
	\includesvg[
		inkscapearea=drawing,
		width=5in,
		angle=-90]{arduino-uno.svg}\par
\end{center}


\vspace*{1cm}
\usubsection{Hardware}



\vspace*{1cm}
\usubsection{Program}

\begin{minted}[]{c}
void setup () {
    pinMode(12, LOW);
}

void loop () {

}
\end{minted}

\vspace*{1cm}
\usubsection{Conclusion}
dfakf

\end{document}