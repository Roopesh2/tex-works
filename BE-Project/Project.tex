\documentclass[12pt, a4paper]{article}
\usepackage{geometry}
\usepackage[utf8]{inputenc}
\usepackage{amsmath}
\usepackage{amsfonts}
\usepackage{mathrsfs}
\usepackage{fancybox,graphicx}
\usepackage{color}
\usepackage[colorlinks]{hyperref}
\usepackage{accents}
\usepackage{cite}
\usepackage{svg}
\usepackage{fancyhdr}
\usepackage{minted}
\usepackage{circuitikz}
\usepackage{standalone}
\usepackage{libertinust1math}
\usepackage{bold-extra}

\pagestyle{fancy}
\fancyhead[C]{}
\fancyfoot[C]{\thepage}
\addtolength{\headwidth}{\marginparsep}
\addtolength{\headwidth}{\marginparwidth}
\renewcommand{\headrulewidth}{0pt}
\renewcommand{\footrulewidth}{.5pt}
\addtolength{\topmargin}{-4.0pt}
\setlength{\headheight}{14.49998pt}

\begin{document}
\begin{titlepage}
	\newgeometry{top=2.5in, bottom=1in, left=1in, right=1in}
	\centering

	{\huge \textbf{Visualisation of Audio using\\ Dot Matrix Display} \par}

	\vspace{1.3cm}

	{\large Submitted by \par}

	\vspace{.5cm}

	{\large \textbf{Roopesh O R}\par}
	{\large \textbf{Prabhath C S}\par}
	{\large \textbf{Pranav Praveen}\par}

	\vspace{3cm}

	\includesvg[inkscapearea=drawing, width=1.5in]{cusat.svg}\par

	\vspace{.1cm}

	{\textbf{Division of Electronics Engineering} \par}
	{\textbf{School of Engineering} \par}
	{\textbf{Cochin University of Science and Technology} \par}
	{\textbf{Kochi - 682022} \par}

	\vspace{.5cm}

	{\textbf{June 2024} \par}

	\vfill

\end{titlepage}

\newgeometry{top=3cm, bottom=3cm, left=2cm, right=2cm}

\newcommand{\usection}[1]{
	\section*{\center \Huge #1}
	\addcontentsline{toc}{section}{\protect\numberline{}#1}
}
\newcommand{\usubsection}[1]{
	\section*{\LARGE #1}
	\addcontentsline{toc}{subsection}{\protect\numberline{}#1}
}

\vspace*{4cm}
\thispagestyle{empty}
\addtocounter{page}{-1}
\usection{Abstract}
\vspace{.5cm}
From simple music player applications to concerts, audio
visualization enhances the ambience and overall experience
of the enjoyer. In this project, we replicate a simpler
version of such systems using an Arduino UNO development
board, 32*8 LED dot matrix display, and a few small components.
An audio signal of our interest is preprocessed and fed
to one of the analogue pins of Arduino. This signal is then
processed using the "ArduinoFFT" library, which splits the
audio into discrete chunks and performs FFT on it. The
resulting frequency spectrum is then again processed and
scaled down to match the width and height of the LED display.
Finally, the resulting waveform is displayed. In the end,
we try to add more customizations such as different "display modes" and effects.


\pagebreak

\usection{Implementation}
\par
\vspace*{1cm}
\usubsection{Introduction}


\vspace*{1cm}
\usubsection{Block Diagram}
\begin{circuitikz} [american voltages]

	\draw
	(-1, 0)
	to [short, *-] (6, 0)
	to [battery1, l_=$+2V$] (6, 2)
	to [R, l_=$R_s$] (6, 3.5)
	to [-] (6, 4)
	to [short, i_={$i_s=2A$}] (4, 4)

	(4.5, 0)
	to [short] (4.5, 1.2)
	to [L, l=$L$] (4.5, 2.3)
	to [short, -*, i={\small $i=3A$}] (4.5, 4)

	(2, 4)
	to [L, l=$L$] (4, 4)

	(2, 4)
	to [R, l_=$R_v$] (.5, 4)
	to [short, -*] (-1, 4)

	(-1, 0) to [open, v^>=$s$] (-1, 4)


	;
\end{circuitikz}


\pagebreak
\usubsection{Arduino UNO}
\vspace{1cm}
\begin{center}
	\includesvg[
		inkscapearea=drawing,
		width=5in,
		angle=-90
	]{arduino-uno.svg}\par
\end{center}

\pagebreak
\usubsection{Hardware}
\includestandalone[width=\textwidth]{hardware/hardware}

\vspace*{1cm}
\usubsection{Program}

\renewcommand{\theFancyVerbLine}{\textcolor[rgb]{0.5,0.5,1.0}{\scriptsize \arabic{FancyVerbLine}}}
\begin{minted}[
	linenos,
	tabsize=2,
	breaklines,
	breakanywhere,
	autogobble,
	fontsize=\footnotesize,
	breakautoindent,
	numbersep=15pt,
	xleftmargin=1cm
]{c}
#include <arduinoFFT.h>
#include <MD_MAX72xx.h>

#define SAMPLES 64
#define HARDWARE_TYPE MD_MAX72XX::FC16_HW
#define MAX_DEVICES 4
#define CLK_PIN 13
#define DATA_PIN 11
#define CS_PIN 10
#define xres 32
#define yres 8

const int buttonPin = 5;
int displaymode = 1;
int previousState = LOW;

double vReal[SAMPLES];
double vImag[SAMPLES];

char data_avgs[xres];

int yvalue;
int displaycolumn, displayvalue;
int peaks[xres];

unsigned long lastDebounceTime = 0;
unsigned long debounceDelay = 200;
const float samplingFrequency = 8000;

int CURRENT_PATTERN[] = { 0, 1, 3, 7, 15, 31, 63, 127, 255 };
const int MODES = 2;
int PATTERNS[MODES][9] = {
  { 0, 1, 3, 7, 15, 31, 63, 127, 255 },// standard pattern
  { 0, 1, 2, 4, 8, 16, 32, 64, 128},   // peak pattern
};


MD_MAX72XX mx = MD_MAX72XX(HARDWARE_TYPE, CS_PIN, MAX_DEVICES);
ArduinoFFT<double> FFT = ArduinoFFT<double>(vReal, vImag, SAMPLES, samplingFrequency);

void setup() {
  // set ADC to free running mode and set pre-scalar to 32 (0xe5)
  ADCSRA = 0b11100101;
  ADMUX = 0b00000000;// A0

  pinMode(buttonPin, INPUT);
  mx.begin();
  // wait to get reference voltage stabilized
  delay(50);
  mx.setColumn(0, 63);
  mx.setColumn(31, 7);
  delay(900);
}

void loop() {
  // Sample collection
  for (int i = 0; i < SAMPLES; i++) {
    // wait for ADC to complete current conversion
    while (!(ADCSRA & 0x10));
    // clear ADIF bit so that ADC can do next operation (0xf5)
    ADCSRA = 0b11110101;
    int value = ADC - 512;
    vReal[i] = value / 8;
    vImag[i] = 0;
  }

  FFT.windowing(vReal, SAMPLES, FFT_WIN_TYP_HAMMING, FFT_FORWARD);
  FFT.compute(vReal, vImag, SAMPLES, FFT_FORWARD);
  FFT.complexToMagnitude(vReal, vImag, SAMPLES);
  int step = (SAMPLES / 2) / xres;

  // shift averages
  for (int i = 0; i < xres; i += step) {
    data_avgs[i] = 0;
    for (int k = 0; k < step; k++) {
      data_avgs[i] = data_avgs[i] + vReal[i + k];
    }
    data_avgs[i] = data_avgs[i] / step;
  }

  // Display
  for (int i = 0; i < xres; i++) {
    data_avgs[i] = constrain(data_avgs[i], 0, 80);
    data_avgs[i] = map(data_avgs[i], 0, 80, 0, yres);
    yvalue = data_avgs[i];

    peaks[i] = peaks[i] - 1;
    if (yvalue > peaks[i])
      peaks[i] = yvalue;
    yvalue = peaks[i];
    displayvalue = CURRENT_PATTERN[yvalue];
    displaycolumn = i;
    mx.setColumn(displaycolumn, displayvalue);
  }
  // check if button pressed to change display mode
  displayModeChange();
}

void displayModeChange() {
  int reading = digitalRead(buttonPin);
  if (
    reading ^ previousState &&
    (millis() - lastDebounceTime) > debounceDelay
  ) {
    displaymode = (displaymode + 1) % MODES;
    for (int i = 0; i <= 8; i++) {
      CURRENT_PATTERN[i] = PATTERNS[displaymode][i];
    }
    lastDebounceTime = millis();
  }
  previousState = reading;
}
\end{minted}

\vspace*{1cm}
\usubsection{Conclusion}

\vspace*{1cm}
\usubsection{References}


\end{document}
