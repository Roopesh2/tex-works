\documentclass[12pt, a4paper]{article}
\usepackage{geometry}
\usepackage[utf8]{inputenc}
\usepackage{mathrsfs}
\usepackage{color}
\usepackage[colorlinks]{hyperref}
\usepackage{svg}
\usepackage{fancyhdr}
\usepackage{minted}
\usepackage{bold-extra}
% \usepackage{mathptmx}
% \usepackage[OT1]{fontenc}
\usepackage{framed}
\usepackage{longtable}
\usepackage{multirow}

\newcommand{\usection}[1]{
	\section*{\center \Huge #1}
	\addcontentsline{toc}{section}{\protect\numberline{}#1}
}
\newcommand{\usubsection}[1]{
	\section*{\LARGE #1}
	\addcontentsline{toc}{subsection}{\protect\numberline{}#1}
}


\pagestyle{fancy}
\fancyhead{}
\fancyfoot{}

\newcommand{\ttbold}[1]{\textbf{\texttt{#1}}}

\fancyhead[C]{}
\fancyfoot[R]{\ttbold{\thepage}}
\fancyfoot[L]{\ttbold{20323085}}
\addtolength{\headwidth}{\marginparwidth}
\renewcommand{\headrulewidth}{0pt}
\renewcommand{\footrulewidth}{.5pt}
\addtolength{\topmargin}{-4.0pt}


\linespread{1.05}
\begin{document}
\begin{titlepage}
	\newgeometry{top=2.5in, bottom=1in, left=1in, right=1in}
	\centering

	{\huge \textbf{Digital Stopwatch} \par}

	\vspace{1.5cm}

	{\large Submitted by \par}

	\vspace{.5cm}

	{\large \textbf{Neha Sethumadhavan (20323080)}\par}
	{\large \textbf{Roopesh O R (20323085)}\par}
	{\large \textbf{Roshna Palatty Santhosh (20323086)}\par}
	{\large \textbf{Sreenandan C K (20323096)}\par}
	{\large \textbf{Merella Jobi (LET B2)}\par}

	\vspace{2cm}

 	\includesvg[inkscapearea=drawing, width=1.5in]{cusat.svg}\par

	\vspace{.1cm}

	{\textbf{Division of Electronics Engineering} \par}
	{\textbf{School of Engineering} \par}
	{\textbf{Cochin University of Science and Technology} \par}
	{\textbf{Kochi - 682022} \par}

	\vspace{.5cm}

	{\textbf{September 2024} \par}

	\vfill

\end{titlepage}
\setlength{\parskip}{5pt}%
\newgeometry{top=3cm, bottom=3cm, left=2.54cm, right=2.54cm}
\thispagestyle{empty}
\addtocounter{page}{-1}
\section*{\Large Abstract}
\vspace{.5cm}
Here we present the design and implementation of a digital stopwatch using a 555 timer as the clock source. The stopwatch displays time in minutes and seconds, utilizing basic digital electronics components such as counters, decoders, and seven-segment displays. The 555 timer is configured in astable mode to generate a clock pulse with a frequency of 1 Hz, serving as the time base for the stopwatch. A series of 60-second counts is accumulated for the seconds, and upon reaching 60, a minute counter increments. These counters are implemented using combination of binary and decade counters, and the output is decoded and displayed on four seven-segment displays, two for minutes and the other two for seconds.
Control functionalities include start, stop, and reset buttons are also present to control the operation of the stopwatch.
\vspace*{2cm}
\section*{\Large Project Estimate}

\begin{tabular}{ |p{5cm}|p{3cm}|p{3cm}|  }
 \hline
 Components & Quantity & Approx. Cost \\
 \hline
7447 - 7 segment decoder & 4 & 80 \\
7408 - AND gate & 2 & 20 \\
7432 - OR gate & 2 & 20 \\
7404 - NOT gates & 1 & 15 \\
7490 - decade counter & 2 & 70 \\
7493 - 4bit counter & 2 & 70 \\
555 timer & 1 & 15 \\
Breadboard & 4-5 & $\sim$200 \\
Capacitor & 4-5 & $\sim$20 \\
Resistors & 35 & $\sim$40 \\
7 segment display & 4 & 120 \\
Potentiometer (100k) & 1 & 20 \\
\hline
\multicolumn{2}{|c|}{Total} & $\sim$ 690 \\

 \hline
\end{tabular}

\end{document}
